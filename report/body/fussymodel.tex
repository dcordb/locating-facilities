\newpage
\subsection{FussyModel}

\begin{flalign}
    &\min f_A(X, Y)& \label{eq:1} \\
    &\min f_B(X, Y)& \label{eq:2}
\end{flalign}
Sujeto a:
\begin{eqnarray}
    f_A(X, Y) &=& \max_{1 \le i \le \abs{A}} \left\{ w(A_i) \cdot \left[ \abs{X(A_i) - X} + \abs{Y(A_i) - Y}  \right] \right\} \nonumber \\
    f_B(X, Y) &=& \abs{X - xca} + \abs{Y - yca} + \abs{X - xcb} + \abs{Y - ycb} \nonumber \\
    (xca, yca) &=& \displaystyle \frac{\sum_{i=1}^{\abs{A}} w(A_i) \cdot (X(A_i), Y(A_i))}{\abs{A}} \label{eq:cma} \\
    (xcb, ycb) &=& \displaystyle \frac{\sum_{j=1}^{\abs{B}} w(B_j) \cdot (X(B_j), Y(B_j))}{\abs{B}} \label{eq:cmb} \\
    x_1 \le &X& \le x_2 \nonumber \\
    y_1 \le &Y& \le y_2 \nonumber
\end{eqnarray}

Notar que \eqref{eq:cma} y \eqref{eq:cmb} son las coordenadas de los centroides de $A$ y $B$ respectivamente.

En este modelo tenemos dos objetivos que queremos satisfacer a la vez. Para esto usamos \href{https://es.wikipedia.org/wiki/Conjunto_difuso}{Teoría de conjuntos difusos} y nos basamos en un resultado obtenido en \cite{Paper}.

La idea general es resolver \eqref{eq:1} y \eqref{eq:2} por separado y luego combinar ambas soluciones en un modelo nuevo que al resolverlo nos dará la solución del modelo original con dos objetivos.

\subsection{Camino a la solución}

En \cite{Paper} exponen un esquema de solución para un problema de optimización con $k$ objetivos. Seguimos ese esquema a continuación.

Denotemos a \eqref{eq:1} como $Z_1$ y a \eqref{eq:2} como $Z_2$. Entonces el primer modelo lo podemos escribir como:

\begin{flalign*}
    & \min Z_1 &
\end{flalign*}
Sujeto a:
\begin{eqnarray*}
    w(A_i) \cdot \left[ \abs{X(A_i) - X} + \abs{Y(A_i) - Y} \right] &\le& Z_1  \quad \forall i \, (1 \le i \le \abs{A}) \\
    x_1 \le &X& \le x_2 \\
    y_1 \le &Y& \le y_2
\end{eqnarray*}

Es decir, $Z_1$ es el máximo que queremos minimizar, por tanto las distancias desde $(X, Y)$ hasta $(X(A_i), Y(A_i))$ tienen que ser menores o iguales a $Z_1$. Denotemos a la solución de este modelo como $(X_1, Y_1)$.

\subsection{Sobre la suma de valores absolutos como restricción}

El modelo anterior tiene restricciones que tienen una suma de dos valores absolutos:

\begin{equation*}
    w(A_i) \cdot \left[ \abs{X(A_i) - X} + \abs{Y(A_i) - Y} \right] \le Z_1  \quad \forall i \, (1 \le i \le \abs{A})
\end{equation*}

Esto lo podemos resolver si introducimos variables $p_i, q_i$ tal que:

\begin{eqnarray*}
    p_i &\ge& \abs{X(A_i) - X} \\
    q_i &\ge& \abs{Y(A_i) - Y}
\end{eqnarray*}

Y por tanto añadimos las siguientes restricciones:

\begin{eqnarray*}
    p_i &\ge& X(A_i) - X \\
    p_i &\ge& X - X(A_i) \\
    q_i &\ge& Y(A_i) - Y \\
    q_i &\ge& Y - Y(A_i) \\
    w(A_i) \cdot \left( p_i + q_i \right) &\le& Z_1
\end{eqnarray*}

\newpage
En el segundo modelo al quitarle los valores absolutos nos queda:

\begin{flalign*}
    & \min Z_2 &
\end{flalign*}
Sujeto a:
\begin{eqnarray*}
    Z_2 &=& t_1 + t_2 + t_3 + t_4 \\
    t_1 &\ge& X - xca \\
    t_1 &\ge& xca - X \\
    t_2 &\ge& Y - yca \\
    t_2 &\ge& yca - Y \\
    t_3 &\ge& X - xcb \\
    t_3 &\ge& xcb - X \\
    t_4 &\ge& Y - ycb \\
    t_4 &\ge& ycb - Y \\ 
    x_1 \le &X& \le x_2 \\
    y_1 \le &Y& \le y_2
\end{eqnarray*}

Denotemos a la solución de este modelo como $(X_2, Y_2)$.

\subsection{Objetivos Difusos}

En \cite{Paper} definen los objetivos difusos de un problema de optimización con $k$ objetivos con la siguiente función de pertenencia:

\begin{equation}
    \mu_k (Z^k(X, Y)) = 
        \begin{cases}
            1   & \text{si} \; Z^k(X, Y) \le L^k \\[4pt]
            \displaystyle \frac{ U^k - Z^k(X, Y) }{ U^k - L^k }  & \text{si} \; L^k \le Z^k(X, Y) \le U^k \\[7pt]
            0   & \text{si} \; Z^k(X, Y) \ge U^k
        \end{cases} \label{eq:mu}
\end{equation}

Donde $U^k$ es la peor cota superior del objetivo $k$ y $L^k$ es la mejor cota inferior del objetivo $k$. En nuestro problema nos quedaría:

\begin{eqnarray*}
    U^1 = \max \left\{ Z^1(X_1, Y_1), Z^1(X_2, Y_2)  \right\} ; &\;& L^1 = Z^1(X_1, Y_1) \\
    U^2 = \max \left\{ Z^2(X_1, Y_1), Z^2(X_2, Y_2)  \right\} ; &\;& L^2 = Z^2(X_2, Y_2) \\
\end{eqnarray*}

Básicamente, $\mu$ le asigna un valor real entre $0$ y $1$ a cada objetivo, que dice que tan bueno es, o dicho de otra forma, que tanto pertenece ese objetivo al conjunto óptimo.

Luego teniendo en cuenta los $L^k, U^k$ y \eqref{eq:mu} estamos en condiciones de formular el modelo final, guiándonos por \cite{Paper}:

\begin{flalign*}
    & \max \lambda &
\end{flalign*}
Sujeto a:
\begin{eqnarray}
    \lambda &\le& \mu_1(Z_1) \nonumber \\
    \lambda &\le& \mu_2(Z_2) \nonumber \\
    w(A_i) \cdot \left[ \abs{X(A_i) - X} + \abs{Y(A_i) - Y} \right] &\le& Z_1  \quad \forall i \, (1 \le i \le \abs{A}) \nonumber \\
    \abs{X - xca} + \abs{Y - yca} + \abs{X - xcb} + \abs{Y - ycb} &=& Z_2 \label{eq:acc} \\
    x_1 \le &X& \le x_2 \nonumber \\
    y_1 \le &Y& \le y_2 \nonumber
\end{eqnarray}

Ponemos \eqref{eq:acc} con valores absolutos para ahorrar en espacio.

Notar que lo que planteamos acá no es más que maximizar el mínimo de las funciones de pertenencia de cada uno de los objetivos. La solución $(X, Y)$ de este modelo lineal sería la solución del modelo inicialmente planteado. 